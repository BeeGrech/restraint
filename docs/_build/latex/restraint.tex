% Generated by Sphinx.
\def\sphinxdocclass{report}
\documentclass[letterpaper,10pt,english]{sphinxmanual}
\usepackage[utf8]{inputenc}
\DeclareUnicodeCharacter{00A0}{\nobreakspace}
\usepackage[T1]{fontenc}
\usepackage{babel}
\usepackage{times}
\usepackage[Bjarne]{fncychap}
\usepackage{longtable}
\usepackage{sphinx}
\usepackage{multirow}


\title{restraint Documentation}
\date{January 24, 2014}
\release{0.1.6}
\author{Bill Peck, Dan Callaghan, Jeff Bastian}
\newcommand{\sphinxlogo}{}
\renewcommand{\releasename}{Release}
\makeindex

\makeatletter
\def\PYG@reset{\let\PYG@it=\relax \let\PYG@bf=\relax%
    \let\PYG@ul=\relax \let\PYG@tc=\relax%
    \let\PYG@bc=\relax \let\PYG@ff=\relax}
\def\PYG@tok#1{\csname PYG@tok@#1\endcsname}
\def\PYG@toks#1+{\ifx\relax#1\empty\else%
    \PYG@tok{#1}\expandafter\PYG@toks\fi}
\def\PYG@do#1{\PYG@bc{\PYG@tc{\PYG@ul{%
    \PYG@it{\PYG@bf{\PYG@ff{#1}}}}}}}
\def\PYG#1#2{\PYG@reset\PYG@toks#1+\relax+\PYG@do{#2}}

\expandafter\def\csname PYG@tok@gd\endcsname{\def\PYG@tc##1{\textcolor[rgb]{0.63,0.00,0.00}{##1}}}
\expandafter\def\csname PYG@tok@gu\endcsname{\let\PYG@bf=\textbf\def\PYG@tc##1{\textcolor[rgb]{0.50,0.00,0.50}{##1}}}
\expandafter\def\csname PYG@tok@gt\endcsname{\def\PYG@tc##1{\textcolor[rgb]{0.00,0.27,0.87}{##1}}}
\expandafter\def\csname PYG@tok@gs\endcsname{\let\PYG@bf=\textbf}
\expandafter\def\csname PYG@tok@gr\endcsname{\def\PYG@tc##1{\textcolor[rgb]{1.00,0.00,0.00}{##1}}}
\expandafter\def\csname PYG@tok@cm\endcsname{\let\PYG@it=\textit\def\PYG@tc##1{\textcolor[rgb]{0.25,0.50,0.56}{##1}}}
\expandafter\def\csname PYG@tok@vg\endcsname{\def\PYG@tc##1{\textcolor[rgb]{0.73,0.38,0.84}{##1}}}
\expandafter\def\csname PYG@tok@m\endcsname{\def\PYG@tc##1{\textcolor[rgb]{0.13,0.50,0.31}{##1}}}
\expandafter\def\csname PYG@tok@mh\endcsname{\def\PYG@tc##1{\textcolor[rgb]{0.13,0.50,0.31}{##1}}}
\expandafter\def\csname PYG@tok@cs\endcsname{\def\PYG@tc##1{\textcolor[rgb]{0.25,0.50,0.56}{##1}}\def\PYG@bc##1{\setlength{\fboxsep}{0pt}\colorbox[rgb]{1.00,0.94,0.94}{\strut ##1}}}
\expandafter\def\csname PYG@tok@ge\endcsname{\let\PYG@it=\textit}
\expandafter\def\csname PYG@tok@vc\endcsname{\def\PYG@tc##1{\textcolor[rgb]{0.73,0.38,0.84}{##1}}}
\expandafter\def\csname PYG@tok@il\endcsname{\def\PYG@tc##1{\textcolor[rgb]{0.13,0.50,0.31}{##1}}}
\expandafter\def\csname PYG@tok@go\endcsname{\def\PYG@tc##1{\textcolor[rgb]{0.20,0.20,0.20}{##1}}}
\expandafter\def\csname PYG@tok@cp\endcsname{\def\PYG@tc##1{\textcolor[rgb]{0.00,0.44,0.13}{##1}}}
\expandafter\def\csname PYG@tok@gi\endcsname{\def\PYG@tc##1{\textcolor[rgb]{0.00,0.63,0.00}{##1}}}
\expandafter\def\csname PYG@tok@gh\endcsname{\let\PYG@bf=\textbf\def\PYG@tc##1{\textcolor[rgb]{0.00,0.00,0.50}{##1}}}
\expandafter\def\csname PYG@tok@ni\endcsname{\let\PYG@bf=\textbf\def\PYG@tc##1{\textcolor[rgb]{0.84,0.33,0.22}{##1}}}
\expandafter\def\csname PYG@tok@nl\endcsname{\let\PYG@bf=\textbf\def\PYG@tc##1{\textcolor[rgb]{0.00,0.13,0.44}{##1}}}
\expandafter\def\csname PYG@tok@nn\endcsname{\let\PYG@bf=\textbf\def\PYG@tc##1{\textcolor[rgb]{0.05,0.52,0.71}{##1}}}
\expandafter\def\csname PYG@tok@no\endcsname{\def\PYG@tc##1{\textcolor[rgb]{0.38,0.68,0.84}{##1}}}
\expandafter\def\csname PYG@tok@na\endcsname{\def\PYG@tc##1{\textcolor[rgb]{0.25,0.44,0.63}{##1}}}
\expandafter\def\csname PYG@tok@nb\endcsname{\def\PYG@tc##1{\textcolor[rgb]{0.00,0.44,0.13}{##1}}}
\expandafter\def\csname PYG@tok@nc\endcsname{\let\PYG@bf=\textbf\def\PYG@tc##1{\textcolor[rgb]{0.05,0.52,0.71}{##1}}}
\expandafter\def\csname PYG@tok@nd\endcsname{\let\PYG@bf=\textbf\def\PYG@tc##1{\textcolor[rgb]{0.33,0.33,0.33}{##1}}}
\expandafter\def\csname PYG@tok@ne\endcsname{\def\PYG@tc##1{\textcolor[rgb]{0.00,0.44,0.13}{##1}}}
\expandafter\def\csname PYG@tok@nf\endcsname{\def\PYG@tc##1{\textcolor[rgb]{0.02,0.16,0.49}{##1}}}
\expandafter\def\csname PYG@tok@si\endcsname{\let\PYG@it=\textit\def\PYG@tc##1{\textcolor[rgb]{0.44,0.63,0.82}{##1}}}
\expandafter\def\csname PYG@tok@s2\endcsname{\def\PYG@tc##1{\textcolor[rgb]{0.25,0.44,0.63}{##1}}}
\expandafter\def\csname PYG@tok@vi\endcsname{\def\PYG@tc##1{\textcolor[rgb]{0.73,0.38,0.84}{##1}}}
\expandafter\def\csname PYG@tok@nt\endcsname{\let\PYG@bf=\textbf\def\PYG@tc##1{\textcolor[rgb]{0.02,0.16,0.45}{##1}}}
\expandafter\def\csname PYG@tok@nv\endcsname{\def\PYG@tc##1{\textcolor[rgb]{0.73,0.38,0.84}{##1}}}
\expandafter\def\csname PYG@tok@s1\endcsname{\def\PYG@tc##1{\textcolor[rgb]{0.25,0.44,0.63}{##1}}}
\expandafter\def\csname PYG@tok@gp\endcsname{\let\PYG@bf=\textbf\def\PYG@tc##1{\textcolor[rgb]{0.78,0.36,0.04}{##1}}}
\expandafter\def\csname PYG@tok@sh\endcsname{\def\PYG@tc##1{\textcolor[rgb]{0.25,0.44,0.63}{##1}}}
\expandafter\def\csname PYG@tok@ow\endcsname{\let\PYG@bf=\textbf\def\PYG@tc##1{\textcolor[rgb]{0.00,0.44,0.13}{##1}}}
\expandafter\def\csname PYG@tok@sx\endcsname{\def\PYG@tc##1{\textcolor[rgb]{0.78,0.36,0.04}{##1}}}
\expandafter\def\csname PYG@tok@bp\endcsname{\def\PYG@tc##1{\textcolor[rgb]{0.00,0.44,0.13}{##1}}}
\expandafter\def\csname PYG@tok@c1\endcsname{\let\PYG@it=\textit\def\PYG@tc##1{\textcolor[rgb]{0.25,0.50,0.56}{##1}}}
\expandafter\def\csname PYG@tok@kc\endcsname{\let\PYG@bf=\textbf\def\PYG@tc##1{\textcolor[rgb]{0.00,0.44,0.13}{##1}}}
\expandafter\def\csname PYG@tok@c\endcsname{\let\PYG@it=\textit\def\PYG@tc##1{\textcolor[rgb]{0.25,0.50,0.56}{##1}}}
\expandafter\def\csname PYG@tok@mf\endcsname{\def\PYG@tc##1{\textcolor[rgb]{0.13,0.50,0.31}{##1}}}
\expandafter\def\csname PYG@tok@err\endcsname{\def\PYG@bc##1{\setlength{\fboxsep}{0pt}\fcolorbox[rgb]{1.00,0.00,0.00}{1,1,1}{\strut ##1}}}
\expandafter\def\csname PYG@tok@kd\endcsname{\let\PYG@bf=\textbf\def\PYG@tc##1{\textcolor[rgb]{0.00,0.44,0.13}{##1}}}
\expandafter\def\csname PYG@tok@ss\endcsname{\def\PYG@tc##1{\textcolor[rgb]{0.32,0.47,0.09}{##1}}}
\expandafter\def\csname PYG@tok@sr\endcsname{\def\PYG@tc##1{\textcolor[rgb]{0.14,0.33,0.53}{##1}}}
\expandafter\def\csname PYG@tok@mo\endcsname{\def\PYG@tc##1{\textcolor[rgb]{0.13,0.50,0.31}{##1}}}
\expandafter\def\csname PYG@tok@mi\endcsname{\def\PYG@tc##1{\textcolor[rgb]{0.13,0.50,0.31}{##1}}}
\expandafter\def\csname PYG@tok@kn\endcsname{\let\PYG@bf=\textbf\def\PYG@tc##1{\textcolor[rgb]{0.00,0.44,0.13}{##1}}}
\expandafter\def\csname PYG@tok@o\endcsname{\def\PYG@tc##1{\textcolor[rgb]{0.40,0.40,0.40}{##1}}}
\expandafter\def\csname PYG@tok@kr\endcsname{\let\PYG@bf=\textbf\def\PYG@tc##1{\textcolor[rgb]{0.00,0.44,0.13}{##1}}}
\expandafter\def\csname PYG@tok@s\endcsname{\def\PYG@tc##1{\textcolor[rgb]{0.25,0.44,0.63}{##1}}}
\expandafter\def\csname PYG@tok@kp\endcsname{\def\PYG@tc##1{\textcolor[rgb]{0.00,0.44,0.13}{##1}}}
\expandafter\def\csname PYG@tok@w\endcsname{\def\PYG@tc##1{\textcolor[rgb]{0.73,0.73,0.73}{##1}}}
\expandafter\def\csname PYG@tok@kt\endcsname{\def\PYG@tc##1{\textcolor[rgb]{0.56,0.13,0.00}{##1}}}
\expandafter\def\csname PYG@tok@sc\endcsname{\def\PYG@tc##1{\textcolor[rgb]{0.25,0.44,0.63}{##1}}}
\expandafter\def\csname PYG@tok@sb\endcsname{\def\PYG@tc##1{\textcolor[rgb]{0.25,0.44,0.63}{##1}}}
\expandafter\def\csname PYG@tok@k\endcsname{\let\PYG@bf=\textbf\def\PYG@tc##1{\textcolor[rgb]{0.00,0.44,0.13}{##1}}}
\expandafter\def\csname PYG@tok@se\endcsname{\let\PYG@bf=\textbf\def\PYG@tc##1{\textcolor[rgb]{0.25,0.44,0.63}{##1}}}
\expandafter\def\csname PYG@tok@sd\endcsname{\let\PYG@it=\textit\def\PYG@tc##1{\textcolor[rgb]{0.25,0.44,0.63}{##1}}}

\def\PYGZbs{\char`\\}
\def\PYGZus{\char`\_}
\def\PYGZob{\char`\{}
\def\PYGZcb{\char`\}}
\def\PYGZca{\char`\^}
\def\PYGZam{\char`\&}
\def\PYGZlt{\char`\<}
\def\PYGZgt{\char`\>}
\def\PYGZsh{\char`\#}
\def\PYGZpc{\char`\%}
\def\PYGZdl{\char`\$}
\def\PYGZhy{\char`\-}
\def\PYGZsq{\char`\'}
\def\PYGZdq{\char`\"}
\def\PYGZti{\char`\~}
% for compatibility with earlier versions
\def\PYGZat{@}
\def\PYGZlb{[}
\def\PYGZrb{]}
\makeatother

\begin{document}

\maketitle
\tableofcontents
\phantomsection\label{index::doc}


Contents:


\chapter{Plugins}
\label{plugins::doc}\label{plugins:welcome-to-restraint-s-documentation}\label{plugins:plugins}
restraint relies on plugins to execute tasks in the correct environment and to check for common errors or sinmply
to provide additional logs for debugging issues.  Here is a typical outline of how plugins are executed:

\begin{Verbatim}[commandchars=\\\{\}]
run\_task\_plugins
 \PYGZbs{}
 10\_bash\_login
 \textbar{}
 20\_unconfined
 \textbar{}
 make run
 \PYGZbs{}
  report\_result
 \PYGZbs{}
  report\_result
run\_task\_plugins
 \PYGZbs{}
 10\_bash\_login
 \textbar{}
 20\_unconfined
 \textbar{}
 run\_plugins \textless{}- completed.d
 \PYGZbs{}
  98\_restore
\end{Verbatim}

The report\_result commands above cause the following plugins to be executed:

\begin{Verbatim}[commandchars=\\\{\}]
run\_task\_plugins
\PYGZbs{}
 10\_bash\_login
 \textbar{}
 20\_unconfined
 \textbar{}
 run\_plugins \textless{}- report\_result.d
 \PYGZbs{}
  01\_dmesg\_check
\end{Verbatim}

These plugins do not run from the task under test.  They run from restraintd process.
This allows for greater flexibility if your task is running as a non-root user since a non-root
user would not be able to inspect some logs and wouldn't be able to clear dmesg log.


\section{Task Run}
\label{plugins:task-run}
Task run plugins are used to modify the environment under which the tasks will execute.
Simply place the executable in /usr/share/restraint/task\_run.d.  The list of files in this directory
will be passed to exec in alphabetical order.

Restraint currently ships with two task run plugins:
\begin{itemize}
\item {} 
10\_bash\_login - invoke a login shell.

\item {} 
20\_unconfined - if selinux is enabled on system run task in unconfined context

\end{itemize}

So the above plugins would get called like so:

\begin{Verbatim}[commandchars=\\\{\}]
exec 10\_bash\_login 20\_unconfined "\$@"
\end{Verbatim}

In order for this to work the task run plugins are required to exec ``\$@'' at the end of the script.
Although task run plugins can't take any arguments they can make decisions based on environment variables.

It should be pointed out that the task run plugins are executed for all other plugins!  This is to ensure
plugins run with the same environment as your task.  When executed under all other plugins the following variable will be defined:

\begin{Verbatim}[commandchars=\\\{\}]
\PYG{n}{RSTRNT\PYGZus{}NOPLUGINS}\PYG{o}{=}\PYG{l+m+mi}{1}
\end{Verbatim}

You can do conditionals based on this so lets create a plugin which will start a tcp capture:

\begin{Verbatim}[commandchars=\\\{\}]
\#Capture tcpdump data from every task
cat \textless{}\textless{} "EOF" \textgreater{} /usr/share/restraint/plugins/task\_run.d/30\_tcpdump
\#!/bin/sh -x

\# Don't run from PLUGINS
if [ -z "\$RSTRNT\_NOPLUGINS" ]; then
  tcpdump -q -i any -q -w \$RUNPATH/tcpdump.cap 2\textgreater{}\&1 \&
  echo \$! \textgreater{} \$RUNPATH/tcpdump.pid
fi

exec "\$@"
EOF
chmod a+x /usr/share/restraint/plugins/task\_run.d/30\_tcpdump
\end{Verbatim}


\section{Report Result}
\label{plugins:report-result}
Every time a task reports a result to restraint these plugins will execute.  Currently restraint only
ships with one plugin:

\begin{Verbatim}[commandchars=\\\{\}]
01\_dmesg\_check
\end{Verbatim}

This plugin checks for the following failure strings:

\begin{Verbatim}[commandchars=\\\{\}]
Oops\textbar{}BUG\textbar{}NMI appears to be stuck\textbar{}cut here\textbar{}Badness at
\end{Verbatim}

But it does then run any matches through an inverted grep which removes the following:

\begin{Verbatim}[commandchars=\\\{\}]
BIOS BUG\textbar{}DEBUG
\end{Verbatim}

This is an effort to reduce false positives.  Both of the above strings can be overridden from each
task by passing in your own FAILURESTRINGS or FALSESTRINGS variables.


\section{Local Watchdog}
\label{plugins:local-watchdog}
These plugins will only be executed if the task runs beyond its expected time limit.  Restraint currently
ships with two plugins:
\begin{itemize}
\item {} 
01\_sysinfo - Collect information about the system, issues sysrq m, t and w. Uploads slabinfo and /var/log/messages.  It will also upload any logs listed in \$TESTPATH/logs2get.

\item {} 
99\_reboot - Simply reboots the system to try and get the system back to a sane state.

\end{itemize}


\section{Completed}
\label{plugins:completed}
These plugins will get executed at the end of every task, regardless if the localwatchdog triggered or not.
The only plugin currently shipped with restraint is:
\begin{itemize}
\item {} 
98\_restore - any files backed up by either rhts-backup or rstrnt-backup will be restored.

\end{itemize}

To finish our tcpdump example from above we can add the following:

\begin{Verbatim}[commandchars=\\\{\}]
\#Kill tcpdump and upload
cat \textless{}\textless{} "EOF" \textgreater{} /usr/share/restraint/plugins/completed.d/80\_upload\_tcpdump
\#!/bin/sh -x

kill \$(cat \$RUNPATH/tcpdump.pid)
rstrnt-report-log -l \$RUNPATH/tcpdump.cap
EOF
chmod a+x /usr/share/restraint/plugins/completed.d/80\_upload\_tcpdump
\end{Verbatim}


\chapter{Indices and tables}
\label{index:indices-and-tables}\begin{itemize}
\item {} 
\emph{genindex}

\item {} 
\emph{modindex}

\item {} 
\emph{search}

\end{itemize}



\renewcommand{\indexname}{Index}
\printindex
\end{document}
